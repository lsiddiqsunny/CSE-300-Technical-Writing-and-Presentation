\documentclass{book}

\usepackage{lipsum}
\usepackage{color}
\usepackage{amsmath}
\usepackage{graphicx}

\newcommand{\testcommand}{Hello}
\newcommand{\anothertestcommand}[2]{Welcome, #2! Mighty #1!!!}
\newcommand{\testingcommand}[3]{hey #1,#2,#3!}

\title{The \textit{First} \LaTeX ~\textbf{Project}}
\author{Mahmudur Rahman}
\date{\today}

\begin{document}

\maketitle

\newpage

\tableofcontents

\newpage

\chapter{Chapter 1}

\section{Introduction}

\subsection{Introductory text}
This is the \textbf{first} CSE 300 class in Jan 2018 term.
This is a new \textcolor{red}{\textbf{line}}.
This is a new line.
This is a new line.
This is a new line.
This is a new line.


\subsection{Introductory text}
\label{sec:intro_text}
This is the \textbf{first} CSE 300 class in Jan 2018 term.
This is a new \textcolor{red}{\textbf{line}}.
This is a new line.
This is a new line.
This is a new line.
This is a new line.

\subsection{More introductory text}
This is the first \textit{CSE 300} class in Jan 2018 term.
This is a new line.
This is a new line.
This is a new line.
This is a new line.
This is a new line. This text is a continuation of Section \ref{sec:intro_text}.

\section{Random Text}

\lipsum

\section{Making Lists}
In this section, we will write various lists.

\subsection{Itemize}

This is an example of itemizing.

\begin{itemize}
    \item CSE 300
    \item CSE 400
\end{itemize}

\subsection{Enumerating}

This is an example of enumerating. Here, items appear under numbers.

\begin{enumerate}
    \item CSE 300
    \item CSE 400
\end{enumerate}

\subsection{Describing}

Example.

\begin{description}
    \item[CSE 300] Tech Writing
    \item[CSE 400] Project and Thesis
\end{description}


\section{Equation}

The famous equation of Newton can be summed up as: $F = ma$. THis is normal text. Following is the expansion of $e$ raised to the power of $x$.

\begin{equation}
    e ^ x = 1 + x + \frac{x ^ 2}{ 2 !} + \frac{x ^ 3}{3!} + \cdot \cdot \cdot ~ \infty
    \label{eqn:e}
\end{equation}

Equation \ref{eqn:e} is the expansion of $e$. This can also be written as:

\begin{equation}
    e ^ x = \sum_{n=0}^{\infty} \frac{x ^ n}{n !}
\end{equation}

This is testing \testcommand. \anothertestcommand{Jon Snow}{Ice King}\\
\testingcommand{sunny}{antara}{tuktuki}

\section{Figures}

Here, we will import figures. Figure  is our figure.


\lipsum

\end{document}
